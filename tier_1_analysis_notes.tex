\documentclass[english]{article}
\hyphenpenalty=10000
% ------------------- Title -------------------
\newcommand{\fullname}{Steven Schaefer}
\newcommand{\email}{stschaef@iu.edu}
\date{}
\title{Tier 1 Analysis}
\author{Put into \LaTeX\ by \\ \fullname \\ \textit{\email}}


% ------------------- Packages -------------------
% Overall formatting, use these pretty much always
\usepackage[utf8]{inputenc}
\usepackage[T1]{fontenc}
\usepackage{tgschola}
\usepackage{titlesec}
\usepackage[margin=1in]{geometry}
\usepackage{pdfpages}

% Tikz stuff. Uncomment when using

% \usepackage{tikz-cd}
% \usepackage{tikz}
% \usepackage{pgfplots}
% \pgfplotsset{compat=1.9}
% \usepgfplotslibrary{fillbetween}

% Math stuff, I'm not entirely sure what all of these do
\usepackage{multicol}
\usepackage{amsmath,amssymb}
\usepackage{amsthm}
\usepackage{varioref}
\usepackage{hyperref}
\usepackage[capitalise]{cleveref}
\usepackage{enumitem} % for autoenumeration of parts
\usepackage{mathtools} % for 'bsmallmatrix' environment

% ------------------- Math Operators -------------------
% mathbb
\DeclareMathOperator{\R}{\mathbb{R}}
\DeclareMathOperator{\Q}{\mathbb{Q}}
\DeclareMathOperator{\N}{\mathbb{N}}
\DeclareMathOperator{\Z}{\mathbb{Z}}
\DeclareMathOperator{\C}{\mathbb{C}}
\DeclareMathOperator{\F}{\mathbb{F}}

\DeclareMathOperator{\Mat}{\text{Mat}}
\DeclareMathOperator{\GCD}{\text{GCD}}
\DeclareMathOperator{\LCM}{\text{LCM}}
\DeclareMathOperator{\Id}{\text{Id}}
\DeclareMathOperator{\id}{\text{id}}
\DeclareMathOperator{\Sym}{\text{Sym}}
\DeclareMathOperator{\Gal}{\text{Gal}}
\DeclareMathOperator{\Aut}{\text{Aut}}
\DeclareMathOperator{\Alt}{\text{Alt}}
\DeclareMathOperator{\Hom}{\text{Hom}}
\DeclareMathOperator{\Tr}{\text{Tr}}
\DeclareMathOperator{\sgn}{\text{sgn}}
\DeclareMathOperator{\rk}{\text{rk}}
\DeclareMathOperator{\Cl}{\text{Cl}}
\DeclareMathOperator{\Log}{\text{Log}}

\renewcommand{\Im}{\text{Im}\ }
\renewcommand{\Re}{\text{Re}\ }

\newcommand{\inv}[1]{#1^{-1}}
\newcommand{\tup}[3]{#1_{#2}, \dots, #1_{#3}}
\newcommand{\into}{\hookrightarrow}
\newcommand{\onto}{\twoheadrightarrow}
\newcommand{\Legendre}[2]{\left( \dfrac{#1}{#2} \right)}

% ------------------- Theorems and Evironments -------------------
% For restarting counter after each section
% \newtheorem{theorem}{Theorem}[section] % the main one 
\newtheorem{theorem}{Theorem} % the main one
\newtheorem{lemma}[theorem]{Lemma}
\newtheorem{proposition}[theorem]{Proposition}

\theoremstyle{definition}
\newtheorem{definition}{Definition}

\newtheorem{problem}[theorem]{Problem}

% \newcounter{problem}
% \newenvironment{problem}[2][Problem \theproblem]
% {\refstepcounter{problem} \hiddenproblem}


\newcounter{subproblem}[theorem]
\renewcommand{\thesubproblem}{(\alph{subproblem})}
\newenvironment{subproblem}[1]{\refstepcounter{subproblem} \begin{trivlist}\item[\hspace{2em} \hskip \labelsep {\bfseries \thesubproblem} \hskip \labelsep {\bfseries #1}]}{\end{trivlist}} 


\crefname{problem}{Problem}{Problems}
\Crefname{problem}{Problem}{Problems}
\crefname{subproblem}{Problem \thetheorem}{}
\Crefname{subproblem}{Problem \thetheorem}{}

\titleformat{\section}
{\normalfont\Large\bfseries}{\thesection}{1em}{}[{\titlerule[0.8pt]}]

% ------------------- Header -------------------
\usepackage{fancyhdr}
\pagestyle{fancy}
\lhead{Tier 1 Analysis}
% \rhead{\textit{stschaef}}

\renewcommand{\labelenumi}{(\roman{enumi})}


% ------------------- Document -------------------
\begin{document}
\begin{minipage}{\textwidth}
    \maketitle
    \begin{abstract}
        I was passed on an extensive collection of handwritten notes -- totaling nearly 600 pages -- for the Tier 1 Analysis Exam at Indiana University. Here I am translating it into \LaTeX, adding my own solutions, and adding any material that I view neceessary. 
        
        I currently do not know who the original author is.
    \end{abstract}
\end{minipage}

\pagebreak

\tableofcontents

\pagebreak

\section{Problems}

\begin{theorem}[Finite Intersection Property \footnote{See \cite{rudin} p.38}]
    \label{thm:finite_intersection}
    If $\{K_{\alpha}\}$ is a collection of compact subsets of a metric space $X$ such that the intersection of every finite subcollection of $\{K_{\alpha}\}$ is nonempty, then $\bigcap K_{\alpha}$ is nonempty. 
\end{theorem}

Note that this may be any collection of compact sets. No countability is assumed.

\begin{proof}[Proof of \cref{thm:finite_intersection}]
    Suppose that $\bigcap K_{\alpha}$ is empty. Fix an arbitary set $K \in \{K_{\alpha}\}$. Because $\bigcap K_{\alpha}$ is empty, for all $x \in K$ we have that $x \in K_{\alpha}^c$ for some $\alpha$. Thus, $\bigcup \left( K_{\alpha}^c \right)$ is an oper cover of $K$. 

    Because $K$ is compact there is a finite subcover, $K_{\alpha_1}^c, \dots, K_{\alpha_n}^c$. That is, $K \subset \bigcup_{i = 1}^n \left( K_{\alpha_i}^c \right)$, so 
    
    \[
        K \cap \left( \bigcap_{i = 1}^n K_{\alpha} \right) = \emptyset
    \]

    This contradicts the hypothesis that the intersection of every finite subcollection of $\{K_{\alpha}\}$ is nonempty. So we conclude that $\bigcap K_{\alpha}$ is nonempty.
\end{proof}

\begin{problem}
    A map $f:\R^m \to \R^n$ is \textit{proper} if it is continuous and $\inv{f}(B)$ is compact for each compact subset $B \subset \R^n$; $f$ is \textit{closed} if it is continuous and $f(A)$ is closed for each closed subset $A \subset \R^m$. 
    
    \begin{subproblem}{}
        \label{prob:proper_is_closed}
        Prove that any proper map $f: \R^m \to \R^n$ is closed.
    \end{subproblem}
    
    
    \begin{subproblem}{}
        \label{prob:one_to_one_closed_is_proper}
        Prove that every one-to-one closed map $f:\R^m \to \R^n$ is proper. 
    \end{subproblem}

    Supposedly this question is from August 1997, but I could find no record of this when combing through previous exams. Note that Tier 1 August 2010, \#2 touches on the same topic of proper maps. 
    
    \begin{proof}[Proof of \cref{prob:proper_is_closed}]
        Let $A \subset \R^m$ be closed. To show that $f(A)$ is closed, it is sufficient to show that $f(A)^c$ is open. 

            Let $q \in f(A)^c$ and consider $\overline{B}_{\epsilon}(q)$ the closed ball of radius $\epsilon$ about $q$. $\overline{B}_{\epsilon}(q)$ is closed and bounded in $\R^n$, so it is compact. Because $f$ is proper, we have that $\inv{f}\left( \overline{B}_{\epsilon}(q) \right)$ is also compact. Define the following sequence of decreasing compact sets,
            
            \[
                U_j = A \cap \inv{f}\left( \overline{B}_{\epsilon_j}(q) \right),
            \]
            where $\epsilon_0 = \epsilon$ and $\epsilon_{j + 1} < \epsilon_j$. Note that each $U_j$ is indeed compact, because $\inv{f}\left( \overline{B}_{\epsilon_j}(q) \right)$ is compact for all $j$ and $A$ is closed. 

            \begin{align*}
                \bigcap U_j
                & = A \cap \bigcap \inv{f}\left( \overline{B}_{\epsilon_j}(q) \right) \\
                & = A \cap \inv{f}\left( q \right) \\
                & = \emptyset
            \end{align*}
            
            The above intersection is empty by construction, as $q \in f(A)^c$.
            
            By \cref{thm:finite_intersection} -- the finite intersection property for compact sets -- because the intersection over all $U_j$'s is empty, there must be some finite subcollection of $\{U_j\}$ with empty intersection. Write this finite subcollection as $U_1, \dots, U_n$. Let $N$  be the smallest such $n$ and note because the $U_j$'s are nested, $\bigcap^N_{j=1} U_j = U_N$. 

            \begin{align*}
                \bigcap^N U_j
                & = U_N \\
                & = A \cap \inv{f}\left( \overline{B}_{\epsilon_N}(q) \right) \\
                & = \emptyset
            \end{align*}

            Thus $f(A) \cap B_{r}(q) = \emptyset$, where $r < \epsilon_N$. So $f(A)^c$ is open and we are done. 
    \end{proof}
    
    \begin{proof}[Proof of \cref{prob:one_to_one_closed_is_proper}]
        TODO
    \end{proof}
\end{problem}

\begin{problem}
    \label{prob:function_of_intersections}
    Let the function $f: \R^n \to \R^n$ satisfy the following two conditions:
    
    \begin{enumerate}
        \item \label{enum:prob:function_of_intersections_1} $f(K)$ is compact whenever $K \subset \R^n$ is compact.
        \item \label{enum:prob:function_of_intersections_2} If $\{K_n\}$ is a decreasing sequence of compact subset of $\R^n$, then 
        
        \[
            f\left(\bigcap_{n = 1}^{\infty} K_n \right) = \bigcap_{n = 1}^{\infty} f(K_n)
        \]
    \end{enumerate}

    Prove that $f$ is continuous.
\end{problem}

\begin{theorem}[Tier 1 August 2015, \#10] \footnote{See \cite{rudin} p.150 and p.516}
    \label{thm:unif_convergence}

    Suppose $K$ is compact, and 
    \begin{enumerate}
        \item \label{enum:thm:unif_convergence_1} $\{f_n\}$ is a sequence of continuous functions on $K$,
        \item \label{enum:thm:unif_convergence_2} $\{f_n\}$ converges pointwise to a continuous function $f$ on $K$,
        \item \label{enum:thm:unif_convergence_3} $f_n(x) \geq f_{n + 1}(x)$ for all $x \in K, n \in \N$.
    \end{enumerate}

    Then $f_n \to f$ uniformly on $K$.
\end{theorem}

\begin{proof}[Proof of \cref{thm:unif_convergence}]
    TODO
\end{proof}

\begin{problem}
    \label{prob:ball_at_origin}
    Let $F: \R^n \to \R^n$ be a differentiable map such that $F(0) = 0$. Assume that 

    \[
      \sum_{j, k = 1}^n \left| {\partial F_j \over \partial x_k} (0) \right|^2 = c < 1.
    \]

    Prove that there is a ball $B \subset \R^n$ with center 0 such that $F(B) \subset B$.
\end{problem}

\begin{proof}[Proof of \cref{prob:ball_at_origin}]
    TODO
\end{proof}

\begin{problem}
    \label{prob:small_ball}
    Suppose that $E \subset \R^n$ is open an that $f: E \to \R^n$ is $C^2$. Suppose also that $f''(x_0)$ is positive definite for some $x_0 \in E$. Prove that there is $r > 0$ such that $f''(x)$ is positive definite for $x \in N_r(x_0)$.
\end{problem}

\begin{proof}[Proof of \cref{prob:small_ball}]
    TODO
\end{proof}

\begin{problem}
    Let $f: \R^n \to \R$ be $C^2$. A point $x \in \R^n$ is a \textit{critical point} of $f$ if all the partial derivatives of $f$ vanish at $x$. A critical point is \textit{nondegenerate} if the $n \times n$ matrix $\left[ {\partial^2 f \over \partial x_i \partial x_j } (x) \right]$ is non-singular. 

    Let $x$ be a nondegenerate critical point of $f$. Prove that there is an open neighborhood of $x$ which contains no other critical points (i.e. the nondegenerate critical points are isolated).
\end{problem}


\pagebreak

\section{Exams Through August 2020}
TODO: Uncomment when finished
% \includepdf[pages=-]{analysis-complete.pdf}

\pagebreak

% ------------------- Bibliography -------------------
\begin{thebibliography}{9}
    \bibitem{rudin}
    Rudin, Walter. \textit{Principles of Mathematical Analysis.} 3d ed, McGraw-Hill, 1976.
\end{thebibliography}

\end{document}